
\begin{enumerate}
\item \textbf{Funcionais} O software AnaclaraBot será incluido no site de uma empresa, com isso, quando algum internauta acessar esse site, poderá iniciar uma conversa. A cada frase que esse internauta enviar, ela responderá conforme a identificação das palavras chaves contidas no texto (a acertude das respostas é de total responsabilidade do \textit{Watson}, juntamente com o treinamento que o administrador do sistema forneceu). A cada interação, as motivadas pelo internauta ou as respondidas pela AnaclaraBot serão armazenadas em um Banco de Dados. A descrição está na Figura \ref{fig:ana}.
\\
\\
\begin{figure}[!ht]
  \centering
      \includegraphics[width=0.92\textwidth]{AnaclaraBot}
  \caption{Descrição de funcionament}
  \label{fig:ana}
\end{figure}


\item \textbf{Não Funcionais} 
\begin{enumerate}
\item O \textit{back-end} estará disponível para as requisições dos internautas, assim como para armazenar as interações, desde que o servidor fornecido por terceiros \textit{Amazon} esteja em pleno funcionamento.

\item Terá uma senha, para que os seus serviços sejam consumidos, essa senha será armazenada em variável de ambiente do sistema operacional.

\item Somente os administradores do sistema terão as senhas de acesso ao servidor.

\item O software será desenvolvido seguindo os padrões de projetos descrita em \hyperref[designpat]{[2]}.

\item O software será desenvolvido para rodar na seguinte configração mínima: 1 CPU, Linux Ubuntu, 2GB de Memória Ram e Acesso a rede internet. Instância T2 Small da \textit{Amazon} \hyperref[amazon]{[3]}.

\item O sistema deverá ser acessado completamente via browser HTTP/HTML e via dispositivos Mobile.

\end{enumerate}


\end{enumerate}